\chapter{Assignment}


% TODO talk about how there are 2 version of the quasi-steady Aerodynamic model
% from A modern course in Aeroelasticity

% TODO talk about how Quasi-steady assumption yields a flutter speed value that
% is approximitely 40% lower than the fully unsteady case! (Blisplinghoff)

% TODO use the k (reduced frequency) value to justify the use of an unsteady
% model (Intro to aeroelasticity and loads)

% TODO Talk about what is circulatory and non-circulatory lift (Intro to
% Aeroelasticity and loads)

% TODO derive the quasi-steady Aerodynamics from Theodorsen (Intro to
% aeroelasticity and loads)

% TODO explain how the presence of the moment coefficient adds a damping
% term in the Liege derivation (Lecture 1 slide 28) This somehow stabilizes the
% system

% TODO Quasi-steady aerodynamics leads to significant modelling errors

% TODO explain why the exact values from Reference 15 were used!

% TODO Use the damping ratio plot for verification of flutter speed with Liege
% Lecture 2

% Quasi-steady Game Plan:
% 1. Derive model from Theodorsen -> Show fucked up results
% 2. Explain how there is a discrepency in the results and the inclusion of
% a pitch damping term has significant effect on the result
% 3. Show the Liege EoM which has a damping term due to the moment coefficient
% 4. In the end, both models have modelling errors -> Use of a Unsteady model
% is justified

\nomenclature[B]{$t$}{Normalized Thickness \nomunit{}}
\nomenclature[B]{$x$}{Normalized Position Along Chord \nomunit{}}
