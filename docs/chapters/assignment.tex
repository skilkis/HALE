\chapter{Assignment Deliverables}
As requested this sole chapter contains the main assignment deliverables
organized such that each section number pertains to the numbering of the tasks.

\section{Typical Section Modelling}
The typical section is a two degree of freedom idealization of a wing, that can
represent the response of the wing in both heave, $h$, and pitch $\theta$.
These degrees of freedom are illustrated below by
\cref{fig:typical_section_convention}. This assignment applies typical section
modelling to the HALE UAV wing with parameters given by \cref{tab:wing_params}.

\begin{figure}[H]
    \centering
    \includegraphics[width=0.85\textwidth]{figures/sign_convention.pdf}
    \caption{Typical Section Dimensions and Sign Convention}
    \label{fig:typical_section_convention}
\end{figure}

\begin{table}[H]
    \centering
    \caption{HALE UAV Wing and Typical Section Parameters \autocite[p.89]{Patil2008}}
    \label{tab:wing_params}
    \begin{tabularx}{\textwidth}{p{0.35\textwidth} C C C} %'L' for Left Aligned, 'C' for Centered, 'R' for Right Aligned
    \toprule
    \textbf{Parameter} & \textbf{Symbol} &\textbf{Value} & \textbf{Units} \\
    \midrule
    Half Span & $b$ & 16 & \si{\meter} \\
    Chord & $c$ & 1 & \si{\meter} \\
    Lift Gradient & $C_{l_\alpha}$ & $2\pi$ & \si{\per\radian}\\
    Airfoil Mass & $m_a$ & 0.75 & \si{\kilogram\per\meter} \\
    Polar Mass Moment of Inertia & $I_\theta^*$ & 0.1 & \si{\kilogram\meter} \\
    Elastic Axis Location & $x_a$ & 0.5 & - \\
    Center of Gravity & $x_{cg}$ & 0.5 & - \\
    Bending Rigidity  & $EI$ & 20000 & \si{\newton\meter\squared} \\
    Torsional Rigidity  & $GJ$ & 10000 & \si{\newton\meter\squared} \\
    \bottomrule
    \end{tabularx}
\end{table}

In essence, this approximation structurally reduces the wing to a cantilevered
beam up until the span-wise position of the typical section, whose properties
are then represented per unit of span. Therefore, since the HALE UAV wing is
rectangular, it is structurally an ideal candidate for modelling with the
typical section. This is because, cross-sectional properties of the wing do not
vary in span, such that they can be represented accurately on a per span basis
without errors.\\

From an aerodynamic point of view, the high-aspect ratio HALE UAV wing is an
ideal candidate to be modelled with the typical section due to the reduced
influence of 3D effects and tip-vortices. This makes it possible to apply the
unsteady 2D aerodynamic model formulation of Theodore Theodorsen
\autocite{ReportNo4961935}.

\section{Spanwise Location Selection}
The location of the typical section must be selected such that it is coincident
with the \textbf{highest loaded section} which occurs at roughly three-quarters
of the half-span \autocite[p.533]{bisplinghoffAeroelasticity1996}.

\section{Tuning of Location Based on Frequency}
\label{sec:tuning}
Even though the typical section generally produces the best results when
placed at three-quarters of the half-span, finding the precise location that
produces the absolute best result requires tuning location based on
the obtained natural frequencies. This section will discuss the structural
modelling and subsequent optimization problem that can be used to locate
the ideal placement of the typical section.

\subsection{Structural Model}
The structural model of the typical section can be derived by using Newtons
second law, and taking the sum force and moments separately in the heave and
pitch degrees of freedom respectively. These are then equated to the freedom
and equating it to the time rate of change of the linear and angular momentum.
Completing this process results in a typical mass spring system with no damping
as linear-structures is assumed, \cref{eq:struct_model}.

\begin{equation}
    \label{eq:struct_model}
    \underbrace{\begin{bmatrix}
        m_a & db \\
        db & I_\theta^* + m_a\left(db\right)^2
    \end{bmatrix}}_{\bm{M_s}}
    \begin{bmatrix}
        \ddot{h} \\ \ddot{\theta}
    \end{bmatrix} +
    \underbrace{\begin{bmatrix}
        K_h & 0 \\
        0 & K_\theta
    \end{bmatrix}}_{\bm{K_s}}
    \begin{bmatrix}
        h \\ \theta
    \end{bmatrix} =
    \begin{bmatrix}
        L \\ M
    \end{bmatrix}
\end{equation}

Note that the lower-right term of the structural mass matrix, $M_s$, represents
the polar mass moment of inertia of the airfoil moved to the elastic axis. This
is why a Steiner term is used to shift the polar mass moment of inertia about
the center of gravity, $I_\theta^*$.\\

Furthermore, the unit bending and torsional stiffness, $K_h$ and $K_\theta$,
can be calculated using \cref{eq:bending_stiffness,eq:torsion_stiffness}
respectively. These are obtained by realizing that a stiffness is the force
generated per displacement of a degree of freedom. Thus, they are re-formulated
from the the generalized beam deflection and angle of twist equations
\autocite[p.865]{hibbelerMechanicsMaterials2011}.\\

Finally, it is important to note that since the typical section modelling
requires quantities expressed on a per unit span basis, these have been divided
by the spanwise position of the typical section. Essentially, this represents
the cantilever beam terminating at the spanwise position of the typical
section, $y_{ts}$, as comprised of an infinite number of sections.

\begin{equation}
    \label{eq:bending_stiffness}
    K_h = \frac{3EI}{L^3 y_{ts}} = \frac{3EI}{y_{ts}^4}
\end{equation}

\begin{equation}
    \label{eq:torsion_stiffness}
    K_\theta = \frac{GJ}{L y_{ts}} = \frac{GJ}{y_{ts}^2}
\end{equation}

\subsection{Determining Natural Frequencies}
Due to the fact that the typical section of the HALE UAV has a center of
gravity that is coincident with the elastic axis, the coupled natural
frequencies are equivalent to the uncoupled natural frequencies. However, it is
desirable to derive a general solution that can also obtain the natural
frequencies of a typical section when this is not the case. The first step in
this process is to set the external aerodynamic forces to zero as the natural
frequencies of the system are those without the presence of external forces
\cref{eq:ext_equal_zero}. As a side note, this is why when doing
Ground-Vibration Tests (GVT) getting the true natural frequencies are so
difficult as the mass of the air causes a slight shift of these frequencies.

\begin{equation}
    \label{eq:ext_equal_zero}
    \bm{M_s\ddot{x}} + \bm{K_s x} = 0
\end{equation}

This equation can be re-arranged to isolate the acceleration terms as given
below by \cref{eq:isolate_accel}.

\begin{equation}
    \label{eq:isolate_accel}
    \bm{M_s^{-1} K_s x} = -\bm{\ddot{x}}
\end{equation}

Next, we can assume a purely harmonic solution of the form
$\bm{x} = \bm{\hat{x}}e^{-i \omega t}$ and differentiate it twice to obtain
\cref{eq:double_diff}. The assumed solution can then be substituted into
\cref{eq:double_diff} to obtain \cref{eq:accel_frequency}.

\begin{equation}
    \label{eq:double_diff}
    \bm{\ddot{x}} = -\bm{\omega^2\hat{x}}e^{-i \bm{\omega} t}
\end{equation}

\begin{equation}
    \label{eq:accel_frequency}
    \bm{\ddot{x}} = -\bm{\omega^2 x}
\end{equation}

Substituting \cref{eq:accel_frequency} into \cref{eq:isolate_accel}
\cref{eq:freq_eig} is obtained which can be seen as an eigen value problem
\cref{eq:eig}. Hence the coupled natural frequencies of the system
$\bm{\omega_n} = \left[w_{n_b}, w_{n_t}\right]^T$.

\begin{equation}
    \label{eq:freq_eig}
    \bm{M_s^{-1} K_s \hat{x}} = \bm{\omega_n^2\hat{x}}
\end{equation}

\begin{equation}
    \label{eq:eig}
    \bm{A\hat{x}} = \bm{\lambda \hat{x}}
\end{equation}

\subsection{Optimized Location of the Typical Section}
Now that a method for determining the natural frequencies in heave and torsion
has been established, the location of the typical section can be tuned using a
numerical optimizer. The aim of the optimizer is to minimize the Residual Sum
of Squares (RSS) between the calculated natural frequency and the exact values
from \autocite{howcroftEfficientAeroelasticBeam2019}. As divergence is dictated
by torsion, the preference is to prioritize a more accurate torsional natural
frequency. However, to verify if this preference is correct three optimization
problems have been defined \cref{eq:h_opt,eq:t_opt,eq:s_opt}, corresponding to
matching the heave, torsion, or both frequencies simultaneously.

\begin{align}
    \text{Minimize:} \hspace{2em}
        f_{h}\left(\eta_{ts}\right) &=
        \left(\omega_{n_{h,ts}} - \omega_{n_{h,exact}}\right)^2
        \label{eq:h_opt}\\[1em]
    \text{Minimize:} \hspace{2em}
        f_{h}\left(\eta_{ts}\right) &=
        \left(\omega_{n_{t,ts}} - \omega_{n_{t,exact}}\right)^2
        \label{eq:t_opt}\\[1em]
    \text{Minimize:} \hspace{2em}
        f_{h}\left(\eta_{ts}\right) &=
        \left(\omega_{n_{h,ts}} - \omega_{n_{h,exact}}\right)^2 +
        \left(\omega_{n_{t,ts}} - \omega_{n_{t,exact}}\right)^2
        \label{eq:s_opt}\\[1em]
    \text{Where:} \hspace{2em} 0 < \eta_{ts} &\leq 1
\end{align}

The result of the torsional frequency optimization, which places a typical
section at $\boxed{y_{ts} = \SI{10.19}{\meter}}$ using \cref{eq:t_opt} is given
below by \cref{tab:torsion_frequencies}. Here, it can be seen that the
torsional frequency is matched exactly.

\begin{table}[H]
    \centering
    \caption{Result of Torsional Frequency Optimization, $y_{ts} =
    \SI{10.19}{\meter}$}
    \label{tab:torsion_frequencies}
    \begin{tabularx}{\textwidth}{p{0.3\textwidth}P{0.095\textwidth}CCC}
    \toprule
    \textbf{Parameter} & \textbf{Symbol} &\textbf{Present Value}
    &\textbf{Ref. Value \autocite{howcroftEfficientAeroelasticBeam2019}}
    & \textbf{Difference} \\
    \midrule
    Torsional Frequency \si{\radian\per\second}
    & $\omega_{n_t}$ & 31.046 & 31.046 & \SI{0.00}{\percent}\\
    Bending Frequency \si{\radian\per\second}
    & $\omega_{n_b}$ & 2.726 & 2.243 & \SI{21.53}{\percent}\\ \bottomrule
    \end{tabularx}
\end{table}

\section{Matching Torsional and Bending Frequencies}
The result of the optimization problems defined previously, visually depcited
by \cref{fig:ts_optimization}, clearly show that it is \textbf{not possible to
match both the torsion and bending natural frequencies} concurrently. This is
because the ideal location of the torsion and heave optimized typical sections
is not coincident. Since both flutter and divergence are more heavily
influenced by the pitch angle, it would reason that selecting the torsionally
optimized typical section location would produce better results. However, one
can also reason that since both bending and heave play a role in the precise
flutter speed calculation, the simultaneously optimized typical section would
produce the most accurate result. Having run the unsteady aerodynamic model for
the latter typical section, it was indeed shown to produce a more accurate
result.

\begin{figure}[H]
    \centering
    \includegraphics[width=0.8\textwidth]{figures/ts_optimization_results.pdf}
    \caption{Squared Residual of Typical Section Optimization Problems}
    \label{fig:ts_optimization}
\end{figure}

\begin{table}[H]
    \centering
    \caption{Typical Section Optimization Results}
    \label{tab:ts_opt_residuals}
    \begin{tabularx}{\textwidth}{p{0.3\textwidth}CCC}
    \toprule
    \textbf{Optimization Target} & $\bm{y_{ts}}$ & $\bm{\eta_{ts}}$ & \textbf{RSS} \\
    \midrule
    Bending Frequency & 11.2294 & 0.7018 & 9.8687e-18 \\
    Torsion Frequency & 10.1858 & 0.6366 & 5.8939e-14 \\
    Both Frequencies & 10.2128 & 0.6383 & 2.2651e-01 \\ \bottomrule
    \end{tabularx}
\end{table}

\begin{table}[H]
    \centering
    \caption{Result of Simultaneous Frequency Optimization, $y_{ts} =
    \SI{10.21}{\meter}$}
    \label{tab:matched_frequencies}
    \begin{tabularx}{\textwidth}{p{0.3\textwidth}P{0.095\textwidth}CCC}
    \toprule
    \textbf{Parameter} & \textbf{Symbol} &\textbf{Present Value}
    &\textbf{Ref. Value \autocite{howcroftEfficientAeroelasticBeam2019}}
    & \textbf{Difference} \\
    \midrule
    Torsional Frequency \si{\radian\per\second}
    & $\omega_{n_t}$ & 30.963 & 31.046 & \SI{0.27}{\percent}\\
    Bending Frequency \si{\radian\per\second}
    & $\omega_{n_b}$ & 2.711 & 2.243 & \SI{20.86}{\percent}\\ \bottomrule
    \end{tabularx}
\end{table}


\section{Steady Divergence Speed}


\subsection{Aerodynamic Model}
Looking at \cref{fig:typical_section_convention} one can see that the only
external force is the lift, $L$. The lift force along with the aerodynamic
moment it causes around the elastic axis, $M$, can be expressed by
\cref{eq:steady_lift,eq:steady_moment} after recognizing that the angle of
attack $\alpha$ is comprised of both a rigid, $\alpha_0$, and flexible
pitch angle, $\theta$, angle \cref{eq:aeroelastic_aoa}.

\begin{equation}
    \label{eq:aeroelastic_aoa}
    \alpha = \alpha_0 + \theta
\end{equation}

\begin{equation}
    \label{eq:steady_lift}
    L = -qSC_{L_\alpha}\left(\alpha_0 + \theta)\right)
\end{equation}

\begin{equation}
    \label{eq:steady_moment}
    M = qSC_{L_\alpha}\left(\alpha_0 + \theta)\right)\cdot
    \left(\frac{1}{2} + a \right)b + M_{ac}
\end{equation}

Assuming that the airfoil is symmetric and thin, the aerodynamic moment at the
aerodynamic center, $M_{ac}$ can be assumed to be equal to zero
\autocite[p.5]{kelleyThinAirfoilTheory2016}. This leaves us with the following
matrix formulation of the aerodynamic model given by \cref{eq:steady_aero}.

\begin{equation}
    \label{eq:steady_aero}
    \begin{bmatrix}
        L \\ M
    \end{bmatrix} =
    q\underbrace{\begin{bmatrix}
        0 & -S C_{L_\alpha} \\
        0 & S C_{L_\alpha}\left(0.5 + a\right)b
    \end{bmatrix}}_{\bm{K_a}}
    \begin{bmatrix}
        h \\ \theta
    \end{bmatrix} +
    \underbrace{\begin{bmatrix}
        -qS C_{L_\alpha} \\
        qS C_{L_\alpha}\left(0.5 + a\right)b
    \end{bmatrix}}_{\bm{B}} \alpha_0
\end{equation}

Note that in \cref{eq:steady_aero} the final rigid angle of attack term,
$\alpha_0$, can be neglected for stability analysis, but is important to keep
for system response simulation as the input matrix, $\bm{B}$, which will be
discussed later.
% TODO reference state-space formulation section

\subsection{Torsional Divergence Speed Calculation}
The name torsional divergence stems from the fact that the divergence phenomena
is influenced only by the pitch degree of freedom. The reason behind this is
that as divergence is a static aeroelastic phenomena, then the lift and moment
on an airfoil is only dependent on the total angle of attack. In other words,
without 3D aerodynamic effects and neglecting changes in density, it does not
matter if the airfoil is 5 or \SI{10}{\meter} above the ground. As such, this
difference in heave displacement does not affect the aerodynamic loads.\\

In order to calculate exact dynamic pressure at which torsional divergence
occurs must then examine the point where the structural stiffness is overcome
by the aerodynamic forces. As the aerodynamic forces increase with an increase
in dynamic pressure, the effective stiffness of the aeroelastic system
reduces with air speed. Furthermore, knowing that negative stiffnesses result
in an unstable structure, the stability boundary is the point where the
aeroelastic stiffness, \cref{eq:aeroleastic_stiffness}, is zero. Thus, any
increase in dynamic pressure beyond this boundary causes the heave and
pitch deflections of the structure to be unbounded
\autocite[p.26]{hulshoffAE4930AeroelasticityCourse2011}.

\begin{equation}
    \label{eq:aeroleastic_stiffness}
    \bm{K_{ae}} = \left(\bm{K_{s}} - q \bm{K_a}\right)
\end{equation}

\cref{eq:aeroleastic_stiffness} above can then be reformulated as a generalized
eigen value problem
\cref{eq:gevp} \autocite[p.652]{zienkiewiczFiniteElementMethod2013}, and then
takes the form given by \cref{eq:q_evp}.

\begin{equation}
    \label{eq:gevp}
    \bm{A}\bm{\hat{x}} = \bm{\lambda}\bm{B}\bm{\hat{x}}
\end{equation}

\begin{equation}
    \label{eq:q_evp}
    \bm{K_s}\bm{\hat{x}} = q_{d}\bm{K_a}\bm{\hat{x}}
\end{equation}

\cref{eq:q_evp} can then be solved numerically and elegantly returns the
divergence dynamic pressure, $q_{d}$, directly since $q_{d} = \lambda$. Note
that the last eigen-value which pertains to the pitch angle, $\theta$, yields
the correct divergence dynamic pressure.

\section{Steady Model Verification}
Now that a method to calculate the divergence speed is obtained, it can be
verified with the HALE UAV reference values. Doing so results in a perfect
match of the divergence speed with $\boxed{U_d=\SI{37.15}{\meter\per\second}}$
for both the present analysis and the reference case.

\subsection{State-Space Formulation \& Response}
As only the stiffness matrices are involved in the divergence speed calculation
the whole aeroelastic system can also be verified by creating state-space
model and simulating the response of the wing. This is achieved by rewriting
the equations of motion to isolate the acceleration terms and inverting
the mass matrix as presented by \cref{eq:isolated_accel}.

\begin{equation}
    \label{eq:isolated_accel}
    \bm{M_{ae}^-1\cdot M_{ae}\ddot{x}} =
    -\bm{M_{ae}^-1\cdot C_{ae}\dot{x}}
    -\bm{M_{ae}^-1\cdot K_{ae}x}
\end{equation}

Next, it is possible to then represent the entire system in a single state
matrix by putting \cref{eq:isolated_accel}, defining an agumented state vector
which includes both the velocity and displacement terms, and making use of the
identity matrix to make the velocities equal to themselves. The result of this
process is given below by \cref{eq:state_matrix}
\autocite[s.4]{andrianneLectureDynamicAeroelasticity}. Additionally, the input
matrix, $\bm{B}$, can be taken as the rigid aerodynamic matrix derived
previously in \cref{eq:steady_aero}. Finally, the output and feed-through
matrices can be set to identity and zero matrices respectively.

\begin{equation}
    \label{eq:state_matrix}
    \begin{bmatrix}
        \ddot{h} \\
        \ddot{\theta} \\
        \dot{h} \\
        \dot{\theta} \\
    \end{bmatrix} =
    \begin{bmatrix}
        \bm{-M_{ae}^-1 C_{ae}} &
        \bm{-M_{ae}^-1 K_{ae}} \\
        \bm{I} & \bm{0}
    \end{bmatrix}
    \begin{bmatrix}
        \dot{h} \\
        \dot{\theta} \\
        h \\
        \theta
    \end{bmatrix}
\end{equation}

With this state-space formulation at hand, the response of the system before
and after divergence is given by \cref{fig:s_simulations}. This clearly shows
how divergence is a static phenomenon as the oscillations are no longer present
at divergence. As a result, of this analysis it can also be concluded that the
mass matrix is behaving correctly.

\begin{figure}[H]
    \centering
    \begin{subfigure}[b]{0.5\textwidth}
        \centering
        \includegraphics[width=0.95\textwidth]{figures/s_simulation_20.pdf}
        \caption{$U =\SI{20}{\meter\per\second}$, $\alpha_0=\SI{0.5}{\radian}$}
        \label{fig:s_simulation_20}
    \end{subfigure}%
    \begin{subfigure}[b]{0.5\textwidth}
        \centering
        \includegraphics[width=0.95\textwidth]{%
            figures/s_simulation_divergence.pdf%
        }
        \caption{$U_d =\SI{37.15}{\meter\per\second}$, $\alpha_0=\SI{0.5}{\radian}$}
        \label{fig:s_simulation_divergence}
    \end{subfigure}
    \caption{Response of Steady Aerodynamic Model Before and at Divergence}
    \label{fig:s_simulations}
\end{figure}

\section{Effect of Spanwise Location on Divergence}
To justify the prioritization of the torsional natural frequency in the typical
section location selection the resultant divergence speeds can be compared.
When using the torsional frequency prioritized typical section, the divergence
speed is $U_d = 37.15$ which matches the reference value exactly. However, when
using the bending frequency prioritized typical section the divergence speed is
$U_d = 33.70$ which is an error of \SI{-9.29}{\percent}. This is not suprising
as divergence is a torsional phenomena. However, this shows that the effect of
moving the typical section outboard is that it decreases the divergence speed.
This makes sense as the unit torsional stiffness reduces with the location of
the typical section as defined by \cref{eq:torsion_stiffness}.

\section{Quasi-Steady Stability Boundaries}
\label{sec:conditions}
To evaluate the stability boundaries of the quasi-steady aeroelastic system one
must recognize that there are three possibilities given below
\autocite[s.11]{andrianneLectureDynamicAeroelasticity}.

\begin{itemize}
    \item \textbf{Case I: Stable System} All eigen values have negative real
    parts
    \item \textbf{Case II: Divergence} At least one eigen-value is positive
    \item \textbf{Case III: Flutter} A pair of complex conjugate
    eigen values has a positive real part
\end{itemize}

These possibilities are in-line with traditional control theory where positive
real eigen values lead to an unstable system. This is due to the real component
of an eigen value is the damping of the system, while the imaginary component
represents the frequency of the motion.

\subsection{Aerodynamic Model}
In order to assess the quasi-steady stability boundaries, the eigen values of
the system must be obtained. However, first the aeroelastic model must be
derived. This assignment requests the use of a simplified version of the
Theodorsen aerodynamic model \autocite{Theodorsen1935}; given by
\cref{eq:theodorsen_lift,eq:theodorsen_moment}.

\begin{equation}
    \label{eq:theodorsen_lift}
    L = -\pi\rho b^2
    \left(U\dot{\theta} + \ddot{h} -ab\ddot{\theta}\right)
    - 2 \pi\rho U b C\left(k\right)
    \left[ U\theta \dot{h} \left(\frac{1}{2} - a \right)b\right]
\end{equation}

\begin{align}
    \label{eq:theodorsen_moment}
    M = &-\pi\rho b^2
    \left(U\left(\frac{1}{2} - a \right)b\dot{\theta} -
    ab\ddot{h} +\left(\frac{1}{8} + a^2\right)b^2\ddot{\theta}\right) \\ \notag
    &+ 2 \pi\rho U b^2 \left(\frac{1}{2} + a \right) C\left(k\right)
    \left[ U\theta \dot{h} \left(\frac{1}{2} - a \right)b\right]
\end{align}

This model makes a distinction between circulatory and non-circulatory
aerodynamic forces. The former are caused by the influence of shed vortices in
the wake. On the other hand, the latter are inertial terms stemming from the
acceleration of air \autocite[p.156]{wrightIntroductionAircraftAeroelasticity}.
What is interesting to observer here is that the circulatory terms are
dependent on an induced angle of attack quantity, $a_i$, caused by a heave and
pitch velocity as shown by \cref{fig:induced_aoa}.

\begin{figure}[H]
    \centering
    \includegraphics[width=0.95\textwidth]{figures/effective_aoa.pdf}
    \caption{Induced Angle of Attack (AoA) due to Pitch and Heave}
    \label{fig:induced_aoa}
\end{figure}

In the derivation of the quasi-steady model, One must be careful as there are
two versions of the quasi-steady model in scientific literature and can lead to
ambiguity in their definition
\autocite[p.95]{dowellModernCourseAeroelasticity2015}. However, in both
definitions the frequency can be assumed to be low enough to neglect, therefore
the derivation of the quasi-steady model from the Theodorsen model involved
setting the reduced frequency, $k$, to zero. Specifically, the reduced
frequency is correlated to the amplitude attenuation and phase-lag caused by
the influence of shed vortices in the flow on the aerodynamic forces
\autocite[p.156]{wrightIntroductionAircraftAeroelasticity}. This means that at
low frequency the phase-lag is not present and the Theodorsen function
$C\left(k\right) = 1$
\autocite[p.158-160]{wrightIntroductionAircraftAeroelasticity}. Making this
simplification and combining acceleration, velocity, and displacement terms
results in the following aerodynamic model presented by \cref{eq:qs_theo}.

\begin{align}
    \label{eq:qs_theo}
    \begin{bmatrix}
        L \\ M
    \end{bmatrix} =
    \pi \rho b^2 \underbrace{\begin{bmatrix}
        -1 & ab \\
        ab & -\left(0.125 + a^2\right)b^2
    \end{bmatrix}}_{\bm{M_a}}
    &\begin{bmatrix}
        \ddot{h} \\ \ddot{\theta}
    \end{bmatrix} \\ \notag
    + \pi \rho U b \underbrace{\begin{bmatrix}
        -2 & -2\left(1 - a\right)b \\
        2\left(a + 0.5\right)b & \left(1 - 2a\right)ab^2
    \end{bmatrix}}_{\bm{C_a}}
    &\begin{bmatrix}
        \dot{h} \\ \dot{\theta}
    \end{bmatrix} \\ \notag
    + \pi \rho U^2 b \underbrace{\begin{bmatrix}
        0 & - 2 \\
        0 & \left(2a + 1\right)b
    \end{bmatrix}}_{\bm{K_a}}
    &\begin{bmatrix}
        h \\ \theta
    \end{bmatrix}
\end{align}

\subsection{Flutter and Divergence Calculation}
To calculate the flutter and divergence boundary, a numerical root finding
algorithm will be used along with the knowledge from the presented divergence
and flutter conditions in \cref{sec:conditions}. However, these require the
eigen-values of the system which can be obtained from the state matrix,
$\bm{E}$. The state matrix for the quasi-steady model now also includes the
damping terms which were previously zero in the steady model. Doing so, the
developed Python code can now retrieve the eigen-values of the system at any
velocity, $U$.\\

To detect divergence, the condition to satisfy is that one of the real
components of the eigen values is positive. As such, the boundary is when one
of the real components is zero. Furthermore, since divergence is a static
phenomena, we can require that the diverging eigen value is only real valued.
The test function for divergence is then given by \cref{eq:div_test}.

\begin{equation}
    \label{eq:div_test}
    \prod_{j=1}^n\Re \left(\lambda_j \in \Re \right) = 0
\end{equation}

Next, to detect flutter the condition to satisfy is that one of the pairs of
complex conjugate eigen values has a positive eigen value. Compared to the case
of testing divergence, the test function for flutter then must require that the
eigen value with a positive real part must have an imaginary component. Once
again the stability boundary is when the real values of the test function are
zero as given by \cref{eq:flut_test}
\autocite[s.19]{andrianneLectureDynamicAeroelasticity}.

\begin{equation}
    \label{eq:flut_test}
    \prod_{j=1}^n\Re \left(\lambda_j \in \Im\right) = 0
\end{equation}

\section{Quasi-Steady Model Verification}
Armed with the test functions derived previously, the flutter and divergence
velocities of the quasi-steady model can now be verified against reference
values. The results of this analysis are presented by
\cref{tab:qs_theo_verification}, where it is clear that the Theodorsen
quasi-steady model performs well for divergence but extremely poorly in
assessing the flutter boundary.

\begin{table}[H]
    \centering
    \caption{Theodorsen Quasi-Steady Aeroelastic Model Result Verification}
    \label{tab:qs_theo_verification}
    \begin{tabularx}{\textwidth}{p{0.275\textwidth}P{0.1\textwidth}CCC}
    \toprule
    \textbf{Parameter} & \textbf{Symbol} &\textbf{Present Value}
    &\textbf{Ref. Value \autocite{howcroftEfficientAeroelasticBeam2019}}
    & \textbf{Difference} \\
    \midrule
    Divergence Speed \si{\meter\per\second}
    & $U_d$ & 37.15 & 37.15 & \SI{0.00}{\percent}\\
    Flutter Speed \si{\meter\per\second}
    & $U_f$ & 13.48 & 32.51 & \SI{-58.53}{\percent}\\
    Flutter Frequency \si{\radian\per\second}
    & $\omega_f$ & 29.04 & 22.37 & \SI{29.81}{\percent} \\ \bottomrule
    \end{tabularx}
\end{table}

This incorrect result can be attributed to how the system is already undamped
in the pitch degree of freedom at in $U=\SI{5}{\meter\per\second}$,
\cref{fig:qs_5} and the damping value of the system seems to already have a
crossing before the detected flutter velocity \cref{fig:qs_theo_frequency}.
Surprisingly, the root-finding algorithm seems to converge on the point where
the damping values diverge. However, examining the response of the system at
the flutter speed, \cref{fig:qs_sim_flutter}, the pitch angle is
growing in amplitude and hence is unstable.

\begin{figure}[H]
    \centering
    \begin{subfigure}[b]{0.5\textwidth}
      \centering
      \includegraphics[width=.95\textwidth]{figures/qs_theo_frequency_diagram.pdf}
      \subcaption{Frequency}
      \label{fig:qs_theo_frequency}
    \end{subfigure}%
    \begin{subfigure}[b]{0.5\textwidth}
      \centering
      \includegraphics[width=.95\textwidth]{figures/qs_theo_damping_diagram.pdf}
      \subcaption{Damping}
      \label{fig:qs_theo_damping}
    \end{subfigure}
    \caption{Theodorsen Quasi-Steady Model Frequency and Damping Diagrams} %Main Caption
    \label{fig:qs_flutter_diagram}
\end{figure}

\begin{figure}[H]
    \centering
    \begin{subfigure}[b]{0.5\textwidth}
        \centering
        \includegraphics[width=0.95\textwidth]{figures/qs_theo_simulation_5.pdf}
        \caption{$U =\SI{5}{\meter\per\second}$, $\alpha_0=\SI{0.5}{\radian}$}
        \label{fig:qs_5}
    \end{subfigure}%
    \begin{subfigure}[b]{0.5\textwidth}
        \centering
        \includegraphics[width=0.95\textwidth]{%
            figures/qs_theo_simulation_flutter.pdf%
        }
        \caption{$U_f =\SI{13.48}{\meter\per\second}$, $\alpha_0=\SI{0.5}{\radian}$}
        \label{fig:qs_sim_flutter}
    \end{subfigure}
    \caption{Response of Quasi-Steady Aerodynamic at Low and Flutter Speed}
    \label{fig:qs_simulations}
\end{figure}

\section{Quasi-Steady Model Limitations}
Not satisfied with the result of the Theodorsen quasi-steady aerodynamic model,
it was deemed necessary to also try the alternative definition from
\autocite[s.2]{andrianneLectureDynamicAeroelasticity} which also includes the
effect of the thin airfoil theory moment coefficient at the elastic axis. This
small change, only effects the lower-right term of the damping matrix presented
by \cref{eq:qs_theo}. The new aerodynamic damping matrix is then given by
\cref{eq:qs_liege}.

\begin{equation}
    \label{eq:qs_liege}
    \bm{C_a} =
    \begin{bmatrix}
        -2 & -2\left(1 - a\right)b \\
        2\left(a + 0.5\right)b & \left(a - a^2 + 0.5\right)b^2
    \end{bmatrix}
\end{equation}

Although this seems like a small change, the aerodynamic damping on the pitch
rate term has a significant impact on the behavior of the system
\autocite[p.168]{wrightIntroductionAircraftAeroelasticity}. With this change,
the flutter speed is increased to $\boxed{U_f =
\SI{24.84}{\meter\per\second}}$. With this more accurate result, it is still
clear that the quasi-aerodynamic model is not suitable to accurately determine
the flutter boundary for this problem. This is evidenced by how quasi-steady
aerodynamic models produce a highly conservative answer that can be as high as
\SI{40}{\percent} lower than those predicted by fully unsteady models.
\autocite[p.568]{bisplinghoffAeroelasticity1996}.

\begin{table}[H]
    \centering
    \caption{Alternative Quasi-Steady Aeroelastic Model Result Verification}
    \label{tab:qs_liege_verification}
    \begin{tabularx}{\textwidth}{p{0.275\textwidth}P{0.1\textwidth}CCC}
    \toprule
    \textbf{Parameter} & \textbf{Symbol} &\textbf{Present Value}
    &\textbf{Ref. Value \autocite{howcroftEfficientAeroelasticBeam2019}}
    & \textbf{Difference} \\
    \midrule
    Divergence Speed \si{\meter\per\second}
    & $U_d$ & 37.15 & 37.15 & \SI{0.00}{\percent}\\
    Flutter Speed \si{\meter\per\second}
    & $U_f$ & 24.84 & 32.51 & \SI{-23.59}{\percent}\\
    Flutter Frequency \si{\radian\per\second}
    & $w_f$ & 25.11 & 22.37 & \SI{12.24}{\percent}\\ \bottomrule
    \end{tabularx}
\end{table}

\begin{figure}[H]
    \centering
    \begin{subfigure}[b]{0.5\textwidth}
      \centering
      \includegraphics[width=.95\textwidth]{figures/qs_liege_frequency_diagram.pdf}
      \subcaption{Frequency}
      \label{fig:qs_liege_frequency}
    \end{subfigure}%
    \begin{subfigure}[b]{0.5\textwidth}
      \centering
      \includegraphics[width=.95\textwidth]{figures/qs_liege_damping_diagram.pdf}
      \subcaption{Damping}
      \label{fig:qs_liege_damping}
    \end{subfigure}
    \caption{Alternative Quasi-Steady Model Frequency and Damping Diagrams} %Main Caption
    \label{fig:qs_liege_flutter_diagram}
\end{figure}

\begin{figure}[H]
    \centering
    \begin{subfigure}[b]{0.5\textwidth}
      \centering
      \includegraphics[width=.95\textwidth]{figures/qs_theo_damping_ratio_diagram.pdf}
      \subcaption{Theodorsen Model}
      \label{fig:qs_theo_damping_ratio}
    \end{subfigure}%
    \begin{subfigure}[b]{0.5\textwidth}
      \centering
      \includegraphics[width=.95\textwidth]{figures/qs_liege_damping_ratio_diagram.pdf}
      \subcaption{Alternative Model}
      \label{fig:qs_liege_damping_ratio}
    \end{subfigure}
    \caption{Comparison of Theodorsen and Alternative Quasi-Steady Damping Ratio} %Main Caption
    \label{fig:qs_damping_ratio_comp}
\end{figure}

\section{Unsteady Stability Boundaries}
To achieve a more accurate flutter speed estimate the effect of the aerodynamic
lag states must be modelled. Although many methods exist in literature, one of
the aims during this assignment was to get a better grasp of the response of a
wing during flight. Therefore, a method which is compatible with a state-space
representation was desired. One such method is to use the indicial function
approximation which transforms the Theodorsen function, $C(k)$, into the time
domain. Thus, it not only allows for the visualization of the system response
in the time domain, but it also increases the code-similarity by
\textbf{allowing use of the flutter and divergence speed calculation method}
discussed previously in \cref{sec:conditions} Unfortunately, since it makes use
of the Wagner function which is an approximation of the Theodorsen function in
the time domain, it is not as exact of a solution as that which can be obtained
by use of a mode-tracking method such as the k or pk methods.

\subsection{Indicial Function Approximation}
The Theodore Function which is composed of transcendental Henkel functions can
be transformed into time-domain using the Wagner function which starts at a
value of 0.5 and slowly progresses to unity as the the airfoil moves further
away from the starting point \cref{eq:wagner}.

\begin{equation}
    \label{eq:wagner}
    \phi\left(t^*\right) = 1
    - \Psi_1 e^{-\epsilon_1 t^*}
    - \Psi_2 e^{-\epsilon_2 t^*}
\end{equation}

The Wagner function is a function of reduced time, $t^*$, which expresses the
number of chord-lengths traversed by the airfoil \cref{eq:reduced_time}. This
represents the phase-lag associated with unsteady aerodynamics as the lift
force gradually increases to its steady-state value.

\begin{equation}
    \label{eq:reduced_time}
    t^* = \frac{Ut}{b}
\end{equation}

The coefficients of the
Wagner function are given below as taken from
\autocite[s.29]{andrianneLectureDynamicAeroelasticity}.

\begin{align*}
      \Psi_1 &= 0.651 \pi \\
      \Psi_2 &= 0.335 \pi \\
      \epsilon_1 &= 0.0455 \\
      \epsilon_2 &= 0.3000 \\
\end{align*}

The derivation of the indicial function approximation is quite complex and
lengthy and for the sake of brevity has been left out of this report. In
essence, involves convoluting the Theodorsen function with the Wagner function
and using the Leibnitz integration rule to obtain so-called ``aerodynamic lag
states'', $w_1, \dots, w_4$, which are a mathematical trick used to produce a
closed-form solution. The result of this derivation is the unsteady aerodynamic
model presented below by \cref{eq:us}, The values of inner aeroelastic
matrices can be obtained from the Python code in \cref{app:listing} or by
referring to \autocite[s.29]{andrianneLectureDynamicAeroelasticity}.

\begin{equation}
    \label{eq:us}
    \begin{bmatrix}
        \ddot{h} \\
        \ddot{\theta} \\
        \dot{h} \\
        \dot{\theta} \\
        w_1 \\
        w_2 \\
        w_3 \\
        w_4 \\
    \end{bmatrix} =
    \begin{bmatrix}
        \bm{-M_{ae}^-1 C_{ae}} &
        \bm{-M_{ae}^-1 K_{ae}} &
        \bm{-M_{ae}^-1 W} \\
        \bm{I} & \bm{0} & \bm{0} \\
        \bm{0} & \bm{W_0} & \dots\\
    \end{bmatrix}
    \begin{bmatrix}
        \dot{h} \\
        \dot{\theta} \\
        h \\
        \theta \\
        w_1 \\
        w_2 \\
        w_3 \\
        w_4 \\
    \end{bmatrix}
\end{equation}

\section{Unsteady Model Verification}
With the fully unsteady indicial function approximation method the calculated
flutter speed of $U_f = \SI{32.77}{\meter\per\second}$ is much more accurate
with a difference of only \SI{0.8}{\percent} with respect to the reference
value. Looking at the damping ratio of the system \cref{fig:us_damping_ratio}
it is also noticeable when the flutter boundary occurs as it is when the system
becomes undamped with a damping ratio of $\xi = 0$
\autocite[s.7]{andrianneLectureDynamicAeroelasticity}. Since this is the only
point in the flight envelope that the damping ratio is equal to zero, the root
finding algorithm to detect the flutter speed is stable.

\begin{table}[H]
    \centering
    \caption{Unsteady Aeroelastic Model Result Verification}
    \label{tab:us_verification}
    \begin{tabularx}{\textwidth}{p{0.275\textwidth}P{0.1\textwidth}CCC}
    \toprule
    \textbf{Parameter} & \textbf{Symbol} &\textbf{Present Value}
    &\textbf{Ref. Value \autocite{howcroftEfficientAeroelasticBeam2019}}
    & \textbf{Difference} \\
    \midrule
    Divergence Speed \si{\meter\per\second}
    & $U_d$ & 37.15 & 37.15 & \SI{0.00}{\percent}\\
    Flutter Speed \si{\meter\per\second}
    & $U_f$ & 32.77 & 32.51 & \SI{0.80}{\percent}\\
    Flutter Frequency \si{\radian\per\second}
    & $w_f$ & 21.87 & 22.37 & \SI{-2.23}{\percent} \\ \bottomrule
    \end{tabularx}
\end{table}

\begin{figure}[H]
    \centering
    \begin{subfigure}[b]{0.5\textwidth}
      \centering
      \includegraphics[width=.95\textwidth]{figures/us_frequency_diagram.pdf}
      \subcaption{Frequency}
      \label{fig:us_frequency}
    \end{subfigure}%
    \begin{subfigure}[b]{0.5\textwidth}
      \centering
      \includegraphics[width=.95\textwidth]{figures/us_damping_diagram.pdf}
      \subcaption{Damping}
      \label{fig:us_damping}
    \end{subfigure}
    \caption{Unsteady Model Frequency and Damping Diagrams} %Main Caption
    \label{fig:us_flutter_diagram}
\end{figure}

\begin{figure}[H]
    \centering
    \includegraphics[width=0.85\textwidth]{figures/us_damping_ratio_diagram.pdf}
    \caption{Unsteady Aeroelastic Model Damping Ratio Diagram}
    \label{fig:us_damping_ratio}
\end{figure}

\section{Unsteady Model Justification}
As it is known that quasi-steady models are not sufficiently accurate for
flutter and gust response calculations when the circulatory lift effects are
significant, it is important to quantify when this takes place. The phase-lag
caused by the circulatory lift takes a maximum value when the reduced frequency
$k=0.3$ \autocite[p.157]{wrightIntroductionAircraftAeroelasticity}. Using the
reduced frequency expression given below by \cref{eq:reduced_frequency}, the
value for this present analysis was found to be $k=0.333$, which signifies
highly unsteady flow and justifies the use of the unsteady aerodynamic model.
This can also be used to explain why the quasi-steady aerodynamic model
produced such poor results with respect to the reference values.

\begin{equation*}
    \label{eq:reduced_frequency}
    k = \frac{\omega b}{U}
\end{equation*}

\section{Modifications to Increase Flutter Speed}
Generally, increasing aeroelastic stability requires increasing the inertial
forces that oppose the excitation of the system. Although, alternatives exist
such as increasing the structural stiffness, those methods typically increase
the mass and complexity of the design
\autocite[p.15]{bontonMeasuringMassProperties2000}. Therefore, the preferred
method is to adjust the mass properties to tune these inertial forces that
oppose the exciting force. A typical method of doing so on control surfaces is
through mass balancing, where the center of gravity is shifted forward to
increase the flutter speed. \cref{fig:aileron_balancing} shows an example of
how such a forward translation of the center of gravity can be achieved through
use of a attached mass. However, to gain a more comprehensive insight into how
Modifications that can be made to a wingbox increase the flutter speed, this
section will conduct a Finite Element Model (FEM) sensitivity analysis and
subsequent optimization to determine an improved design.

\begin{figure}[H]
    \centering
    \includegraphics[width=0.6\textwidth]{figures/control_surface_balancing.pdf}
    \caption{
        Control Surface Balancing Options
        \autocite{bingelisControlSurfaceBalancing}
    }
    \label{fig:aileron_balancing}
\end{figure}

\subsection{Idealized Wingbox Geometry Used in Analysis}
The idealized wingbox geometry selected for this analysis presented below
by \cref{fig:wingbox_geometry}. The wingbox takes the form of the symmetric
airfoil profile on the top and bottom skin connected by straight spars at
the front and rear. Such a geometry is typically used in conceptual wingbox
design and sizing \autocite[p.540]{ajajConceptualWingboxWeight2013}.

\begin{figure}[H]
    \centering
    \includegraphics[width=0.85\textwidth]{figures/wingbox_drawing.pdf}
    \caption{Idealized Wingbox Geometry}
    \label{fig:wingbox_geometry}
\end{figure}

\subsection{Initial Wingbox Derivation}
As the goal of the assignment is to understand the effect of structural
modifications on the stability boundaries, the exact design of the wingbox is
not significant. With that said, it is still required to somehow derive a
baseline design that enables the exploration of the structural sensitivities.
Furthermore, since information on the HALE UAV wing is not sufficient to
constrain the wingbox design space to a singular design there are two
approaches that are possible. The first approach is to define realistic
material properties of the wingbox and find a suitable geometry to end up with
the same divergence and flutter speeds as the refernce values.\\

Alternatively, the second approach, that is used in this present analysis, is
to define a suitable geometry first and then derive a hypothesized material
based on the inertial properties of the geometry. Therefore, the first step in
this method is to find a suitable geometry for the wingbox. A bounded
optimization problem is set-up to minimize the offset between the elastic axis
and center of gravity with respect to their reference values,
\cref{eq:geometric_opt}. The initial values and bounds of the wingbox design
vector $v_{wb}$ are given by \cref{tab:design_bounds}.

\begin{equation}
    \label{eq:geometric_opt}
    \text{Minimize:} \hspace{2em}
        f\left(v_{wb}\right) =
        \left(x_a - a_{a,exact}\right)^2 +
        \left(x_{cg} - a_{cg,exact}\right)^2
\end{equation}

\begin{table}[H]
    \centering
    \caption{Wingbox Design Vector $v_{wb}$ Initial Value and Bounds}
    \label{tab:design_bounds}
    \begin{tabularx}{\textwidth}{p{0.35\textwidth}CCCC} %'L' for Left Aligned, 'C' for Centered, 'R' for Right Aligned
    \toprule
    \textbf{Variable} & \textbf{Symbol} & \textbf{Initial} &\textbf{Lower}
    & \textbf{Upper}\\
    \midrule
    Front Spar Location & $x_{fs}$ & 0.250 & 0.100 & 0.450 \\
    Rear Spar Location & $x_{rs}$ & 0.700 & 0.550 & 0.950\\
    Front Spar Thickness & $t_{fs}$ & 0.005 & 0.001 & 0.050\si{-} \\
    Rear Spar Thickness & $t_{rs}$ & 0.040 & 0.001 & 0.050 \\
    Skin Thickness & $t_{skin}$ & 0.010 & 0.001 & 0.050 \\ \bottomrule
    \end{tabularx}
\end{table}

As HALE UAV aircraft typically have thick airfoils
\autocite{hallissyHighfidelityAeroelasticAnalysis2011}, a NACA 0015 symmetric
profile is chosen as the outer moldline of the wingbox. Due to this curved
airfoil geometry, the section properties are then retrieved by a Finite Element
Model (FEM) using the \texttt{sectionproperties} Python package
\autocite{vanleeuwenSectionproperties}. In order to speed up the convergence of
the optimizer, the skin thickness was not allowed to go below
\SI{1}{\milli\meter}. Otherwise, the number of cells required to accurately
represent the geometry grow too large, and iteration times quickly climb to
upwards of a few minutes. Running this optimization then produces the wingbox
geometry given by \cref{fig:geometric_wingbox_geom} with convergence history
given by \cref{fig:geometric_wingbox_hist}. The final values the typical
section locations as well as the design vector is provided
\cref{tab:matched_ts_values,tab:geometric_wbox_values}.

\begin{figure}[H]
    \centering
    \includegraphics[width=0.95\textwidth]{figures/geometric_wbox_opt_centroids.pdf}
    \caption{FEM Mesh and Geometry of Geometrically Matched Wingbox}
    \label{fig:geometric_wingbox_geom}
\end{figure}

\begin{figure}[H]
    \centering
    \includegraphics[width=0.85\textwidth]{figures/geometric_wbox_opt_history.pdf}
    \caption{Convergence History of Geometrically Matched Wingbox}
    \label{fig:geometric_wingbox_hist}
\end{figure}

\begin{table}[H]
    \centering
    \caption{Final Values of the Matched Typical Section Locations}
    \label{tab:matched_ts_values}
    \begin{tabularx}{\textwidth}{CCCC} %'L' for Left Aligned, 'C' for Centered, 'R' for Right Aligned
    \toprule
    \textbf{Parameter} & \textbf{Calculated Value} & \textbf{Target Value}
    & \textbf{Difference} \\
    \midrule
    $x_a$ & 0.4998 & 0.5000 & \SI{-0.04}{\percent}\\
    $x_{cg}$ & 0.4999 & 0.5000 & \SI{-0.02}{\percent}\\ \bottomrule
    \end{tabularx}
\end{table}

\begin{table}[H]
    \centering
    \caption{Geometrically Matched Wingbox Geometric Parameters}
    \label{tab:geometric_wbox_values}
    \begin{tabularx}{\textwidth}{p{0.35\textwidth}CCC} %'L' for Left Aligned, 'C' for Centered, 'R' for Right Aligned
    \toprule
    \textbf{Parameter} & \textbf{Symbol} & \textbf{Value} &\textbf{Units}\\
    \midrule
    Front Spar Location & $x_{fs}$ & 0.234 & \si{-} \\
    Rear Spar Location & $x_{rs}$ & 0.684 & \si{-} \\
    Front Spar Thickness & $t_{fs}$ & 0.005 & \si{-} \\
    Rear Spar Thickness & $t_{rs}$ & 0.039 & \si{-} \\
    Skin Thickness & $t_{skin}$ & 0.009 & \si{-} \\ \bottomrule
    \end{tabularx}
\end{table}

\subsection{Sensitivity Analysis}
The reason for conducting such a relatively high fidelity analysis was to gain
insight into how the design variables of the idealized wingbox geometry affect
the stability boundary of the wing. A way for engineers to gain a better
understanding of a problem is to then perform a sensitivity analysis. The most
direct sensitivity to measure is how the wingbox variables affect the
stability boundaries, \cref{fig:wingbox_aero_sens}. Here, one can see that the
highest sensitivity is measured by the location of the front and rear spars
which drastically change the location of the elastic axis and center of
gravity. If the front spar is moved backward both the divergence and flutter
speeds decrease. On the other hand if the rear spar is moved backward the
divergence and flutter speeds increase.\\

To understand this behavior better it is important to relate the geometric
wingbox parameters to how they influence the typical section parameters,
\cref{tab:wingbox_ts_sensitivity}, and those in turn to their effect on the
\cref{fig:ts_aero_sens}. Here we can see that the front spar location
has the highest influence on decreasing the mass of the airfoil which in turn
should increase the flutter speed. However, moving the front spar backward
causes center of gravity to shift aft as well. Looking at
\cref{fig:ts_aero_sens}, this shows us that the center of gravity position has
the highest effect on the flutter speed, which confirms the aforementioned
effect of mass balanced ailerons.

\begin{table}[H]
    \centering
    \caption{Response of Typical Section to a 1\% Increase in Wingbox Variables}
    \label{tab:wingbox_ts_sensitivity}
    \begin{tabularx}{\textwidth}{CCCCCC} %'L' for Left Aligned, 'C' for Centered, 'R' for Right Aligned
    \toprule
    \textbf{Response} & $\uparrow\bm{x_{fs}}$ & $\uparrow\bm{x_{rs}}$ & $\uparrow\bm{t_{fs}}$ &
    $\uparrow\bm{t_{rs}}$ & $\uparrow\bm{t_{skin}}$ \\
    \midrule
    $\bm{m_a}$ &
    \SI{-0.435}{\percent} &
    \SI{0.447}{\percent} &
    \SI{-0.049}{\percent} &
    \SI{0.165}{\percent} &
    \SI{0.522}{\percent} \\
    $\bm{I_\theta^*}$ &
    \SI{-0.274}{\percent} &
    \SI{0.579}{\percent} &
    \SI{0.000}{\percent} &
    \SI{0.015}{\percent} &
    \SI{0.067}{\percent} \\
    $\bm{x_a}$ &
    \SI{0.252}{\percent} &
    \SI{0.169}{\percent} &
    \SI{-0.151}{\percent} &
    \SI{0.045}{\percent} &
    \SI{0.086}{\percent} \\
    $\bm{x_{cg}}$ &
    \SI{0.211}{\percent} &
    \SI{0.569}{\percent} &
    \SI{-0.028}{\percent} &
    \SI{0.077}{\percent} &
    \SI{-0.071}{\percent} \\
    $\bm{EI}$ &
    \SI{-0.874}{\percent} &
    \SI{0.072}{\percent} &
    \SI{-0.274}{\percent} &
    \SI{-0.244}{\percent} &
    \SI{0.441}{\percent} \\
    $\bm{GJ}$ &
    \SI{-1.392}{\percent} &
    \SI{2.937}{\percent} &
    \SI{-0.004}{\percent} &
    \SI{0.076}{\percent} &
    \SI{0.341}{\percent} \\ \bottomrule
    \end{tabularx}
\end{table}

\begin{figure}[H]
    \centering
    \includegraphics[width=0.85\textwidth]{figures/wingbox_aeroelastic_sensitivity.pdf}
    \caption{Response of Aeroelastic Speeds to a \SI{1}{\percent} Increase in
    Wingbox Variables}
    \label{fig:wingbox_aero_sens}
\end{figure}

\begin{figure}[H]
    \centering
    \includegraphics[width=0.85\textwidth]{figures/typical_section_aeroelastic_sensitivity.pdf}
    \caption{Response of Aeroelastic Speeds to a \SI{1}{\percent} Increase in
    Typical Section Variables}
    \label{fig:ts_aero_sens}
\end{figure}

\subsection{Proposed Modification}
In the highly coupled and non-linear design space presented previously, it is
difficult for a human to make decisions that will guarantee that the flutter
frequency matches the divergence speed. However, looking at
\cref{fig:wingbox_aero_sens} it is clear that one way to achieve this
is to increase the skin thickness which increases the flutter speed more than
the divergence speed. Additionally, this modification must be coupled with
a decrease in rear spar location which will decrease the flutter speed less
that a more severe decrease to the divergence speed.

\section{Effect of Modifications on Divergence}

As discussed, the proposed modification is to \textbf{increase the skin
thickness} while \textbf{move the rear spar forward}. By fine tuning the amount
by which these two modifications are applied it should be possible to increase
the flutter speed while keeping the divergence speed constant.

\section{Aeroelastic Wingbox Optimization}

To answer the question of whether it is possible to match the divergence and
flutter speed exactly, a second optimization problem is defined making use of
the objective function given below by \cref{eq:aeroelastic_opt}. The resultant
geometry of the aeroelastic optimization of the wingbox along with the
convergence history is provided below by \cref{fig:aero_wingbox_geom} and
\cref{fig:aero_wingbox_hist} respectively.

\begin{equation}
    \label{eq:aeroelastic_opt}
    \text{Minimize:} \hspace{2em}
        f\left(x\right) =
        \left(U_d - U_{d,exact}\right)^2 +
        \left(U_f - U_{d,exact}\right)^2 \\[1em]
\end{equation}

Looking at the final achieved values for the flutter and divergence speeds,
\cref{tab:matched_wbox_values} it is clear that indeed it is possible to match
the flutter speed to the divergence speed. Furthermore, even though the
optimizer utilized more than just the proposed modifications to achieve this
result, the optimizer did indeed increase the skin thickness from the initial
value of 0.009 to 0.017. Additionally, the optimizer also heavily modified the
location and thickness of the rear spar by shifting it to the most frontal
position of $x_{fs} = 0.5$. For comparison purposes, the final values of the
aeroelastically optimized wingbox are presented by
\cref{tab:aeroelastic_wbox_values,tab:aeroelastic_ts_values} below which can be
contrasted with the geometrically optimized wingbox and initial typical section
respectively.

\begin{figure}[H]
    \centering
    \includegraphics[width=0.95\textwidth]{figures/aeroelastic_wbox_opt_centroids.pdf}
    \caption{FEM Mesh and Geometry of Aeroelastically Optimized Wingbox}
    \label{fig:aero_wingbox_geom}
\end{figure}

\begin{table}[H]
    \centering
    \caption{Final Values of the Matched Flutter and Divergence Speeds}
    \label{tab:matched_wbox_values}
    \begin{tabularx}{\textwidth}{CCCC} %'L' for Left Aligned, 'C' for Centered, 'R' for Right Aligned
    \toprule
    \textbf{Parameter} & \textbf{Calculated Value} &\textbf{Target
    Value} & \textbf{Difference} \\
    \midrule
    $U_d$ & 37.12 & 37.15 & \SI{-0.08}{\percent}\\
    $U_f$ & 37.14 & 37.15 & \SI{-0.03}{\percent}\\ \bottomrule
    \end{tabularx}
\end{table}

\begin{figure}[H]
    \centering
    \includegraphics[width=0.85\textwidth]{figures/aeroelastic_wbox_opt_history.pdf}
    \caption{Convergence History of Aeroelastically Optimized Wingbox}
    \label{fig:aero_wingbox_hist}
\end{figure}

\begin{table}[H]
    \centering
    \caption{Aeroelastically Optimized Wingbox Geometric Parameters}
    \label{tab:aeroelastic_wbox_values}
    \begin{tabularx}{\textwidth}{p{0.35\textwidth}CCC} %'L' for Left Aligned, 'C' for Centered, 'R' for Right Aligned
    \toprule
    \textbf{Parameter} & \textbf{Symbol} & \textbf{Value} &\textbf{Units}\\
    \midrule
    Front Spar Location & $x_{fs}$ & 0.217 & \si{-} \\
    Rear Spar Location & $x_{rs}$ & 0.550 & \si{-} \\
    Front Spar Thickness & $t_{fs}$ & 0.008 & \si{-} \\
    Rear Spar Thickness & $t_{rs}$ & 0.041 & \si{-} \\
    Skin Thickness & $t_{skin}$ & 0.017 & \si{-} \\ \bottomrule
    \end{tabularx}
\end{table}

\begin{table}[H]
    \centering
    \caption{Aeroleastically Optimized Wingbox Typical Section Parameters}
    \label{tab:aeroelastic_ts_values}
    \begin{tabularx}{\textwidth}{p{0.35\textwidth}CCC} %'L' for Left Aligned, 'C' for Centered, 'R' for Right Aligned
    \toprule
    \textbf{Parameter} & Symbol & \textbf{Value} &\textbf{Units}\\
    \midrule
    Airfoil Mass & $m_a$ & 0.958 & \si{\kilogram\per\meter}\\
    Polar Mass Moment of Inertia & $I_\theta^*$ & 0.095 & \si{\kilogram\meter}\\
    Elastic Axis Location & $x_a$ & 0.439 & \si{-} \\
    Center of Gravity & $x_{cg}$ & 0.409 & \si{-} \\
    Bending Rigidity & $EI$ & 26809 & \si{\newton\meter\squared} \\
    Torsional Rigidity & $GJ$ & 7542 & \si{\newton\meter\squared} \\ \bottomrule
    \end{tabularx}
\end{table}
